\documentclass[fleqn,a4paper,12pt]{article}
\usepackage[top=1in, bottom=1in, left=1in, right=1in]{geometry}


\usepackage{filecontents}

\begin{filecontents}{\jobname.bib}
\end{filecontents}


\title{深度學習應用——作業四報告}
\author{B03902072 江廷睿}
\date{}

\usepackage{listings}

\usepackage{amsmath}
\usepackage{amssymb}

\usepackage{graphicx}
\usepackage[margin=1cm]{caption}
\usepackage{subcaption}
\usepackage{float}

\usepackage{mathspec}
\setmainfont{Noto Serif CJK TC}
% \setmathsfont(Digits,Latin,Greek)[Numbers={Lining,Proportional}]{DejaVu Math TeX Gyre}
\newfontfamily\ZhFont{Noto Serif CJK TC}
\newfontfamily\SmallFont[Scale=0.8]{Droid Sans}
% \newfontfamily\SmallSmallFont[Scale=0.7]{Noto Serif CJK}
\usepackage{fancyhdr}
\usepackage{lastpage}
\pagestyle{fancy}
\fancyhf{}
\rhead{B03902072\ZhFont{江廷睿}}
\lhead{深度學習應用——作業四}
\cfoot{\thepage / \pageref{LastPage}}
\XeTeXlinebreaklocale "zh"

\renewcommand\tablename{表}
\renewcommand\figurename{圖}

\begin{document}
\maketitle
\thispagestyle{fancy}

\section{Model Description}

\subsection{模型架構}

\subsubsection*{圖片標籤的前處理}

圖片部份的前處理有:把原圖縮小到 $64 \times 64$ ,並把 $[0, 255]$ 的顏色線性對應到 $[-1, +1]$。標籤則是只取頭髮顏色與眼睛顏色相關的標籤,並把頭髮髮顏色跟眼睛顏色各自 one-hot encode 後連接起來作為 generator 與 discriminator 的 condition。

\subsubsection*{Generator}

把 condition 的向量跟 100 維從標準常態分佈取樣出來的雜訊相接,通過 filter 個數為 512、256、128、64、3 ,尺寸為 $4 \times 4$ , stride 為 $2$ 的 deconvolution layer ,除了最後一層以外,每一層都使用 Batch Normalization 與激發函數 RuLU  。最後一層則是使用 $\tanh$ 作為激發函數,產生 $64 \times 64$ ,通道數為 $3$ 的圖像。

\subsubsection*{Discriminator}

把輸入通過 filter 個數分別為 32、64、128、256,尺寸為 $4 \times 4$, stride 為 $2$ ,的 convolution layer ,得到 $4 \times 4$ ,通道數為 $256$ 的張量。接著把 condition 的向量複製 $16$ 份,接到 convolution layer 產生的張量後。接著再把得到的張量通過個數為 128、128 ,尺寸為 $1 \times 1$ 與 $4 \times 4$ 的 convolution layer 。其中除了最後一層以外,在每一層後都使用 Leaky ReLU 作為激發函數。

\subsection{Generator 跟 Discriminator 的目標函數}

令 $x$ 為隨機取樣的真實圖片,$c$ 為那張圖片的 condition , $c'$ 為隨機取樣的錯誤 condition ,$G$ 為 generator 函數、$D$ 為 discriminator 函數 $\alpha$ 為從 $[0, 1)$ 的均勻分布取樣的隨機數。generator 的目標函數為

\begin{equation*}
  \max D(G(c), c)
\end{equation*}

discriminator 的目標函數為

\begin{equation*}
  \max D(x, c) -\max\{D(G(c), c), D(x, c')\} - \lambda (\lVert \nabla D(\alpha x + (1 - \alpha) G(c)) \rVert_2 - 1)^2
\end{equation*}

這裡 $\lambda$ 使用 10 。

\subsection{其餘超參數}

\begin{itemize}
\item Optimizer: RMSprop
\item Learning Rate: $0.0002$
\item 此外每更新 5 次 discriminator 才更新一次 generator 。
\end{itemize}


\section{How do you improve your performance}

\subsection{使用 Improved W-GAN}

相較於 W-GAN , Improved W-GAN 不對 discriminitor 的參數的最 weight-clipping ,而是在產生的圖片與真的圖片之間隨機內插一點,並在目標函數中減少 generator 在該點的梯度的 L2-norm 。

\subsection{Discriminator 的目標函數}

原先使用的損失函數為

\begin{equation*}
  \max D(x, c) - 0.5 D(G(c), c) - 0.5 D(x, c') - \lambda (\lVert \nabla D(\alpha x + (1 - \alpha) G(c)) \rVert_2 - 1)^2  
\end{equation*}

但有可能是因為 discriminator 可以把 $D(G(c), c)$ 降得很低,以至於忽略了 $D(G(c), c)$ ,所以產生的圖像不太考慮 condition 。因此若是把目標函數改成

\begin{equation*}
  \max D(x, c) -\max\{D(G(c), c), D(x, c')\} - \lambda (\lVert \nabla D(\alpha x + (1 - \alpha) G(c)) \rVert_2 - 1)^2
\end{equation*}

就可以強迫 discriminator 必須要同時能夠分辨出假的圖像還要能分辨出 condition 才能增加整體目標函數的數值。

\section{Experiment settings and observation}

\subsection{使用 Improved W-GAN}

使用 improved W-GAN 相較於一般的 GAN ,可以產生更清晰的圖片。

\subsection{Discriminator 的目標函數}

在使用更改後的目標函數後,產生出來的圖片顯然更與 condition 相關。

\end{document}
