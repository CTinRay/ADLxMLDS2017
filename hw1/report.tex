\documentclass[fleqn,a4paper,12pt]{article}
\usepackage[top=1in, bottom=1in, left=1in, right=1in]{geometry}



\title{深度學習應用——作業一報告}
\author{B03902072 江廷睿}
\date{}

\usepackage{listings}

\usepackage{amsmath}
\usepackage{amssymb}

\usepackage{graphicx}
\usepackage[margin=1cm]{caption}
\usepackage{subcaption}
\usepackage{float}


\usepackage{mathspec}
\setmainfont{Noto Serif CJK TC}
% \setmathsfont(Digits,Latin,Greek)[Numbers={Lining,Proportional}]{DejaVu Math TeX Gyre}
\newfontfamily\ZhFont{Noto Serif CJK TC}
\newfontfamily\SmallFont[Scale=0.8]{Droid Sans}
% \newfontfamily\SmallSmallFont[Scale=0.7]{Noto Serif CJK}
\usepackage{fancyhdr}
\usepackage{lastpage}
\pagestyle{fancy}
\fancyhf{}
\rhead{B03902072\ZhFont{江廷睿}}
\lhead{深度學習應用——作業一}
\cfoot{\thepage / \pageref{LastPage}}

\renewcommand\tablename{表}

\begin{document}
\maketitle
\thispagestyle{fancy}

\section{Model description}
\label{sec:model}

\subsection{Preprocess}

只有用到 MFCC 這個特徵。特徵中的每個維度都有各自進行縮放、平移,使得訓練資料中 39 維中的每一個維度的平均值都是 0 ,標準差都是 1 。每句話作為一筆訓練資料。

\subsection{參數與最佳化方式}

\begin{itemize}
\item 批的大小:32
\item 最佳化方式:Adam
\item 學習率:0.001
\end{itemize}


\subsection{RNN}

\begin{itemize}
\item Dropout,丟失比例 0.2
\item 隱藏狀態維度為 256 的 Bi-directional LSTM
\item Dropout,丟失比例 0.2
\item 隱藏狀態維度為 256 的 Bi-directional LSTM
\item Dropout,丟失比例 0.2
\item 隱藏狀態維度為 128 的 Bi-directional LSTM
\item 對於每個訊框,各自線性轉換成 39 維的分數,並做 softmax 
\end{itemize}

\subsection{RNN + CNN}

\begin{itemize}
\item 核大小 7 , 64 個過濾器的一維卷積層,補齊方式為 same 
\item 批標準化層
\item 線性整流函數
\item 以下重複 4 次:
  \begin{itemize}
  \item 核大小 7 , 64 個過濾器的一維卷積層,補齊方式為 same
  \item 批標準化層
  \item 線性整流函數
  \item Dropout,丟失比例 0.1
  \item 核大小 7 , 64 個過濾器的一維卷積層,補齊方式為 same
  \item 與上上上上層的輸入相加
  \item 批標準化層
  \item 線性整流函數
  \item Dropout,丟失比例 0.1
  \end{itemize}
\item 隱藏狀態維度為 128 的 Bi-directional LSTM
\item 對於每個訊框,各自線性轉換成 39 維的分數,並做 softmax 
\end{itemize}


\subsection{Best}

\begin{itemize}
\item 核大小 7 , 64 個過濾器的一維卷積層,補齊方式為 same 
\item 批標準化層
\item 線性整流函數
\item 以下重複 10 次:
  \begin{itemize}
  \item 核大小 7 , 64 個過濾器的一維卷積層,補齊方式為 same
  \item 批標準化層
  \item 線性整流函數
  \item Dropout,丟失比例 0.2
  \item 核大小 7 , 64 個過濾器的一維卷積層,補齊方式為 same
  \item 與上上上上層的輸入相加
  \item 批標準化層
  \item 線性整流函數
  \item Dropout,丟失比例 0.2
  \end{itemize}
\item 對於每個訊框,各自線性轉換成 39 維的分數,並做 softmax 
\end{itemize}


\section{How to improve your performace}

\subsection{Dropout}

Dropout 就是在訓練過程中,隨機將上一層的部份輸出設成 0 的技術。使用的主要理由是實驗上在驗證資料中的確能延遲過度擬合,並帶來比較高的準確度。

\subsection{加深網路}

實驗結果顯示,無論是 RNN 或 CNN ,把網路加深都有助於模型的準確度。

\subsection{使用 LSTM}

實驗結果顯示,從 GRU 換成 LSTM 也能帶來一些提昇,推測可能是因為 LSTM 比 GRU 有更強的表現能力。

\subsection{殘差網路與批標準化}

參考《Deep Residual Learning for Image Recognition》,這裡使用了殘差的架構,據稱可以增進訓練深度網路的效果。批標準化也是論文中建議使用的方法。

\section{Experimental results and settings}

\subsection{RNN 與 RNN + CNN 與 CNN 的比較}

表 \ref{tab:rnn_rnncnn_cnn_cmp} 三個模型的表現,他們的架構如第 \ref{sec:model} 節,其餘設定都相同。從表格中可以看到,雖然 RNN + CNN 的架構有最高的訊框準確度(所有訊框的預測結果中,正確的比例),但 edit distance 卻是三者中最差的。純粹 RNN 的架構表現在三個架構的中間,而純粹的 CNN 則有最好的 edit distance 。


\begin{table}
  \centering
  \begin{tabular}[H]{ | c | c | c | c | }
    \hline
    & RNN & RNN + CNN & RNN \\
    \hline
    訊框準確度(驗證集) & 0.8008 & 0.8101 & 0.7989 \\
    \hline
    Edit Distance(驗證集) & 10.1299 & 10.0825 & 9.5440 \\
    \hline
    Edit Distance(Public Leaderboard) & 10.6384 & 10.9496 & 9.7853 \\  
    \hline
  \end{tabular}
  \caption{RNN、RNN + CNN、CNN 三種架構的比較。}
  \label{tab:rnn_rnncnn_cnn_cmp}
\end{table}

\subsection{LSTM 與 GRU 的比較}

也有實驗過將 LSTM 換成 GRU 的 RNN 模型。但 GRU 模型的各種表現都比不上 LSTM ,訊框準確度為 0.7947 ,public leaderboard 上的 edit distance 也只有 11.8814 。

\subsection{不同深度的 CNN}

實驗顯示不管有沒有用 RNN ,使用較深 CNN 都可以有比較好的效果。以第 1 節提到的 RNN + CNN 架構為例, 如果只使用 9 層的 CNN ,訊框準確度只有 0.7949 ,edit distance 也只到 11.5593 。


\clearpage
\appendix

\section{使用的套件}

\begin{itemize}
\item bleach==1.5.0
\item editdistance==0.3.1
\item html5lib==0.9999999
\item Markdown==2.6.9
\item numpy==1.13.3
\item pandas==0.20.3
\item protobuf==3.4.0
\item python-dateutil==2.6.1
\item pytz==2017.2
\item six==1.11.0
\item tensorflow-gpu==1.3.0
\item tensorflow-tensorboard==0.1.8
\item tqdm==4.19.2
\item Werkzeug==0.12.2
\end{itemize}



\end{document}
