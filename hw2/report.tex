\documentclass[fleqn,a4paper,12pt]{article}
\usepackage[top=1in, bottom=1in, left=1in, right=1in]{geometry}


\usepackage{filecontents}

\begin{filecontents}{\jobname.bib}
  @inproceedings{DBLP:conf/ijcai/SongGGLZS17,
    author    = {Jingkuan Song and
      Lianli Gao and
      Zhao Guo and
      Wu Liu and
      Dongxiang Zhang and
      Heng Tao Shen},
    title     = {Hierarchical {LSTM} with Adjusted Temporal Attention for Video Captioning},
    booktitle = {Proceedings of the Twenty-Sixth International Joint Conference on
      Artificial Intelligence, {IJCAI} 2017, Melbourne, Australia, August
      19-25, 2017},
    pages     = {2737--2743},
    year      = {2017},
    url       = {https://doi.org/10.24963/ijcai.2017/381},
    doi       = {10.24963/ijcai.2017/381},
    timestamp = {Tue, 15 Aug 2017 14:48:05 +0200},
    biburl    = {http://dblp.org/rec/bib/conf/ijcai/SongGGLZS17},
    bibsource = {dblp computer science bibliography, http://dblp.org}
  }
  @inproceedings{DBLP:conf/iccv/VenugopalanRDMD15,
    author    = {Subhashini Venugopalan and
      Marcus Rohrbach and
      Jeffrey Donahue and
      Raymond J. Mooney and
      Trevor Darrell and
      Kate Saenko},
    title     = {Sequence to Sequence - Video to Text},
    booktitle = {2015 {IEEE} International Conference on Computer Vision, {ICCV} 2015,
      Santiago, Chile, December 7-13, 2015},
    pages     = {4534--4542},
    year      = {2015},
    crossref  = {DBLP:conf/iccv/2015},
    url       = {https://doi.org/10.1109/ICCV.2015.515},
    doi       = {10.1109/ICCV.2015.515},
    timestamp = {Thu, 15 Jun 2017 21:45:06 +0200},
    biburl    = {http://dblp.org/rec/bib/conf/iccv/VenugopalanRDMD15},
    bibsource = {dblp computer science bibliography, http://dblp.org}
  }
  @proceedings{DBLP:conf/iccv/2015,
    title     = {2015 {IEEE} International Conference on Computer Vision, {ICCV} 2015,
      Santiago, Chile, December 7-13, 2015},
    publisher = {{IEEE} Computer Society},
    year      = {2015},
    url       = {http://ieeexplore.ieee.org/xpl/mostRecentIssue.jsp?punumber=7407725},
    isbn      = {978-1-4673-8391-2},
    timestamp = {Thu, 12 May 2016 10:41:46 +0200},
    biburl    = {http://dblp.org/rec/bib/conf/iccv/2015},
    bibsource = {dblp computer science bibliography, http://dblp.org}
  }

\end{filecontents}


\title{深度學習應用——作業二報告}
\author{B03902072 江廷睿}
\date{}

\usepackage{listings}

\usepackage{amsmath}
\usepackage{amssymb}

\usepackage{graphicx}
\usepackage[margin=1cm]{caption}
\usepackage{subcaption}
\usepackage{float}

\usepackage{mathspec}
\setmainfont{Noto Serif CJK TC}
% \setmathsfont(Digits,Latin,Greek)[Numbers={Lining,Proportional}]{DejaVu Math TeX Gyre}
\newfontfamily\ZhFont{Noto Serif CJK TC}
\newfontfamily\SmallFont[Scale=0.8]{Droid Sans}
% \newfontfamily\SmallSmallFont[Scale=0.7]{Noto Serif CJK}
\usepackage{fancyhdr}
\usepackage{lastpage}
\pagestyle{fancy}
\fancyhf{}
\rhead{B03902072\ZhFont{江廷睿}}
\lhead{深度學習應用——作業二}
\cfoot{\thepage / \pageref{LastPage}}
\XeTeXlinebreaklocale "zh"

\renewcommand\tablename{表}

\begin{document}
\maketitle
\thispagestyle{fancy}

\section{Model description}
\label{sec:model}

\subsection{參數與最佳化方式}

\begin{itemize}
\item 批的大小:64
\item 最佳化方式:Adam
\item 學習率:0.001
\end{itemize}

\subsection{架構描述}

參考 ~\cite{DBLP:conf/ijcai/SongGGLZS17} 提出的 hLSTMat 架構以及一些誤解,實作了如下的架構:

\subsubsection{編碼器}

使用助教提供的,對於每個訊框,用 VGG16 抽取的特徵:

\begin{equation}
  V = \{ v_1, v_2, \cdots, v_{80} \}
\end{equation}

\subsubsection{解碼器}

\begin{itemize}
\item 底層的 LSTM 以上一個字的 embedding 作為輸入,參考原論文, LSTM 的大小是 512 。 \footnote{這邊的 $\tanh$ 是參考原作者的原始碼,在原論文中沒有出現。}
  \begin{equation}
    \begin{aligned}
      h^\vee_0, c^\vee_0 =& \tanh([W^{ih}; W^{ic}] \mathrm{Mean}({v_1, v_2, \cdots, v_{80}})) \\
      h^\vee_t, c^\vee_t =& \mathrm{LSTM}(\mathrm{embedding}(y_t), h^\vee_{t-1}, c^\vee_{t-1})
    \end{aligned}
  \end{equation}
  其中 embedding 與 $W^{ih}, W^{ic}$ 都是要學的參數, 參考作者的原始碼使用 embedding 維度 512 。
  
\item 上層的 LSTM以下層的輸出 $h^\vee$ 作為輸入,參考原論文, LSTM 的大小也是 512:
  \begin{equation}
    \begin{aligned}
      h^\wedge_t, c^\wedge_t =& \mathrm{LSTM}(h^\vee, h^\wedge_{t-1}, c^\wedge_{t-1})      
    \end{aligned}
  \end{equation}
  
\item \label{attention} Attention:在每個時間點 $t$ 使用下層 LSTM 的輸出 $h_t^\wedge$ 計算出權重
  \begin{equation}
    \alpha = \mathrm{softmax}(w^T \tanh(W_a h^\wedge_t + U_a V + b_a))
  \end{equation}
  其中 $w, W_a, U_a, b_a$ 都是要學的參數。原論文沒有說明這之中的維度,原作者的原始碼不太合理的使用影片特徵的維度,所以這部份我隨意的設成 128 。
  然後算出 attention \footnote{原論文中 attention 前面多乘了一個 $\frac{1}{n}$} 
  \begin{equation}
    a = \sum_i \alpha_i v_i 
  \end{equation}
  
\item 輸出層:將上層 LSTM 的輸出 $h^\wedge$ 與 attention $a$ 相接後,輸入一個兩層的神經網路,並用 softmax 把輸出轉化成預測每個字的機率:
  \begin{equation}
    P(y) = \mathrm{softmax}(U_p \tanh(W_p[h^\wedge; a] + b_p) + d)
  \end{equation}
  其中中間層的維度參考原作者可怕的原始碼,同樣使用 512。
\item Dropout:在上述的模型中,對 $h^\wedge, h^\vee, a$ 以及多層神經網路的中間層使用 dropout ,丟失比例參考論文使用 0.5 。

\end{itemize}

\section{Attention Mechanism}


\subsection{How do you implement attention mechanism?}

請參考 \ref{attention} 中 Attention 的部份。

\subsection{Compare and analyze the results of models with and without attention mechanism. }

這個架構在拿掉 attention (把 attention 換成 0 向量)後的 BLEU@1 分數只有 0.66 ,而若是加上 attention ,分數則有 0.70 。另外 S2VT \cite{DBLP:conf/iccv/VenugopalanRDMD15} 這個沒有 attention 的架構也只有 0.66 。因此,顯然 attention 的確能增進模型的效果。

\section{How to improve your performace}

\subsection{特徵標準化}

把影片的特徵的每個維度都減去他們的平均值並除以他們的標準差,使得每個維度的平均值為 0 ,標準差為 1 。理論上這可以讓梯度下降的過程更順利,並加速訓練。實際上這的確能使模型的 bleu score 上升得更快。

\subsection{兩層的 LSTM 架構}

根據論文,這種架構的理念是希望上層的 LSTM 可以專住在學習語言模型,而下層的 LSTM 則是專住在處理影片的資訊。實際上而言,因為這個架構沒有使用 RNN 編碼影片,所以訓練速度會快上許多,而且 BLEU@1 分數能達到 0.70 ,因此最終決定採取這個架構。

\section{Experimental results and settings}

\subsection{Schedule Sampling}

實驗發現使用 schedule sampling 沒辦法有效的增加 BLEU 分數,此外使用 Schedule Sample 容易使模型產生不合文法的句子,像是「A man is a a」,因此最後沒有使用 schedule sampling。

\subsection{改變 Embedding 大小與 Hidden Layer 的維度}

這裡嘗試著改變 embedding 維度與輸出神經網路的 hidden layer 的維度,但因為原論文(程式碼)中 embedding 大小與 hidden layer 維度都相同,因此只有測試他們使用相同維度的狀況。在嘗試過 256、 512、768 的大小之後,發現這三組參數的 BLEU@1 分數都在 0.69 到 0.70 之間。 

\subsection{改變 Dropout Rate}

雖然最後的模型是使用 dropout rate 0.5 訓練出來的,但實驗證明在維度等於 512 的情況下,沒有 dropout 或是 dropout rate 等於 0.25 的模型也可以達到 BLEU@1 0.70 的分數。

\bibliographystyle{ieeetr}
\bibliography{\jobname}

\end{document}
